from prompt import constant
import generativeai, generativeai_for_photo
from flask import Flask, request, jsonify, render_template
from flask_cors import CORS, cross_origin
import os
from werkzeug.utils import secure_filename
from PIL import Image  
import uuid

app = Flask(__name__)

CORS(app, resources={r"/*": {"origins": "*"}})

# Set the folder to store uploaded files
app.config['UPLOAD_FOLDER'] = 'uploads'

@app.route('/')
def index():
    # return render_template('fasion_tips_with_skin_tone_photo.html')
    #  return render_template('low_your_periodpain.html')
    #  return render_template('pregnency_tracker.html')
     return render_template('skin_care.html')

@app.route('/low_your_period_pain', methods=['POST'])
def low_your_period_pain():
    data = request.get_json()  
    symptoms = data.get('symptoms')  
    if symptoms:
        try:
            prompt = constant.prompt_for_low_your_period_pain + symptoms
            answer = generativeai.callingai(prompt)  
            return jsonify({'Excercises': answer})  
        except Exception as e:
            return jsonify({'error': str(e)}), 500  
    return jsonify({'error': 'Invalid symptoms'}), 400

@app.route('/pregnency_tracker', methods=['POST'])
def pregnency_tracker():
    data = request.get_json()  
    month = data.get('month')
    if month:
        try:
            prompt = constant.prompt_for_pragnency_traker.format(month=month)
            answer = generativeai.pregnency_traker(prompt)  
            return jsonify({'tips': answer})  
        except Exception as e:
            return jsonify({'error': str(e)}), 500  
    return jsonify({'error': 'Invalid symptoms'}), 400

@app.route('/fasion_tips_with_skin_tone_photo', methods=['POST'])
@cross_origin(supports_credentials=True)
def fasion_tips_with_skin_tone_photo():
    if 'image' not in request.files:
        return jsonify({"message": "No image uploaded"}), 400

    file = request.files['image']
    if file.filename == '':
        return jsonify({"message": "No selected file"}), 400

    # Generate a unique filename
    filename = f"{uuid.uuid4()}_{secure_filename(file.filename)}"
    file_path = os.path.join(app.config['UPLOAD_FOLDER'], filename)
    
    # Ensure the upload directory exists
    if not os.path.exists(app.config['UPLOAD_FOLDER']):
        os.makedirs(app.config['UPLOAD_FOLDER'])

    # Save the file to the server
    file.save(file_path)

    # Call your AI function with the uploaded image
    prompt = constant.prompt_for_fashion_with_photo
    ai_response = generativeai_for_photo.generate_ai_response(prompt, file_path)

    return jsonify({"message": "Image uploaded successfully", "ai_response": ai_response})

@app.route('/hair_care', methods=['POST'])
@cross_origin(supports_credentials=True)
def hair_care():
    data = request.get_json()
    hair_type = data.get('hair_type')
    weather = data.get('weather')
    
    if hair_type or weather:
        try:
            prompt = constant.prompt_for_hair_care.format(hair_type=hair_type, weather=weather)
            answer = generativeai.callingai(prompt)  
            return jsonify({'tips': answer})  
        except Exception as e:
            return jsonify({'error': str(e)}), 500  
    return jsonify({'error': 'Invalid symptoms'}), 400

@app.route('/skin_care', methods=['POST'])
@cross_origin(supports_credentials=True)
def skin_care():
    if 'image' not in request.files:
        return jsonify({"message": "No image uploaded"}), 400

    file = request.files['image']
    prompt = request.form.get('prompt')
    skin = request.form.get('skin')

    if file.filename == '':
        return jsonify({"message": "No selected file"}), 400

    # Generate a unique filename
    filename = f"{uuid.uuid4()}_{secure_filename(file.filename)}"
    file_path = os.path.join(app.config['UPLOAD_FOLDER'], filename)
    
    # Ensure the upload directory exists
    if not os.path.exists(app.config['UPLOAD_FOLDER']):
        os.makedirs(app.config['UPLOAD_FOLDER'])

    # Save the file to the server
    file.save(file_path)
    
    # Process the prompt and skin data as needed
    # Example response (customize this as needed)
    response_data = {
        "message": "File uploaded successfully",
        "file_path": file_path,  # Optional, don't expose paths in production
        "tips": f"Based on your prompt '{prompt}' and skin type '{skin}', here are some tips..."
    }
    
    return jsonify(response_data)

if __name__ == '__main__':
    app.run(debug=True)
